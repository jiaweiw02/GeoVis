\documentclass[9pt,twocolumn,twoside]{opticajnl}
\journal{opticajournal} % use for journal or Optica Open submissions

% See template introduction for guidance on setting shortarticle option
\setboolean{shortarticle}{true}
% true = letter/tutorial
% false = research/review article

% ONLY applicable for journal submission shortarticle types:
% When \setboolean{shortarticle}{true}
% then \setboolean{memo}{true} will print "Memorandum" on title page header
% Otherwise header will remain as "Letter"
% \setboolean{memo}{true}

\usepackage{lineno}
\linenumbers % Turn off line numbering for Optica Open preprint submissions.

\title{GeoVis: an interactive map to visualize common climate indicators}

\author[1,2]{Muhammad Alafifi}
\author[1,3]{Jiawei Wu}

\affil[1]{Computer Science Department, Rensselaer Polytechnic Institute, 1999 Burdett Ave, Troy NY, 12180}

\affil[2]{email: alafim@rpi.edu}
\affil[3]{email: wuj22@rpi.edu}

\begin{abstract}
    Geovis is a map rendering of the United States that aims to provide 
    climate enthusiasts or curious individuals with an interactive webpage 
    to visualize common climate indicators such as wind speed, precipitation, 
    and temperature. With the increase in greenhouse gas emissions around the world, 
    it is crucial to visualize such trends that leverages ones understand in climate 
    change. To compile this project, we utilized datasets from \href{https://www.kaggle.com/datasets}{Kaggle}.
    Our visualization garnered much important feedback, with many feedback being appreciative of our minimalistic and intuitive design. 
    We found that many testers were curious at our web application because they were found to 
    be clicking on multiple things for the sake of visualizing the data.
\end{abstract}

\setboolean{displaycopyright}{false} % Do not include copyright or licensing information in submission.

\begin{document}

\maketitle

\section{Introduction and Background}

We are both climate enthusiasts, which is why we decided to pursue a 
project that was closely related to visualizing climate. Our intention 
of creating this project was to provide a platform for individuals to 
visualize common climate indicators such as wind speed, precipitation, 
and temperature. We believe that with the increase in greenhouse gas 
emissions around the world, it is crucial to visualize such trends that 
leverages ones understand in climate change. Creating a map visualization 
was what we believed to be the most intuitive way to visualize data, we were 
inspired by \href{http://hint.fm/wind/}{wind fm} and \href{https://earth.nullschool.net/}
{earth nullschool}. These sites provide the user with great flexibility in 
manipulating and showing specific climate indicators, something that we took 
inspiration in creating our own.

\section{Data Collection}

To compile this project, we primarily utilized datasets of type csv. 
There are two types of data that we had to process, one is for every 
airport, and another for every county.

\subsection{Airport Data}

Utilizing airports was the most intuitive way to gather weather data. Initially we had planned to use large global datasets, such as \href{https://www.kaggle.com/datasets/noaa/noaa-icoads?select=icoads_core_2017}{NOAA ICOADS}, but we found it expensive, given the size of the dataset being 31 million rows and 75 columns. Not only that, but the points that were collected over the United States were so abundant, that it was difficult to visualize and also hindered our efforts to make it interactive. Instead, we pivoted over to a dataset that was more manageable, which was the \href{https://www.kaggle.com/datasets/srinathnanduri97/weather-data-for-265-airports-in-the-united-states}{weather data for 265 airports in the United States}. This dataset was much more manageable, with 265 airports. Airports resembled a cleaner visualization as well, as some states have at most 3 major airports, making it simple to visualize the difference.

\textbf{Processing the data} was by no means simple. There wasn't a dataset that was "perfect" for our project, so we had to look for multiple datasets. The datasets we found was always missing some key features. Our airport weather data, while it was useful in providing us with the weather data, it was missing the latitude and longitude of the airports. We had to find \href{https://www.kaggle.com/datasets/nancyalaswad90/us-airports}{another} dataset that provided us with the latitude and longitude of the airports. We had to merge these two datasets together to create a dataset that had both the weather data and the latitude and longitude of the airports. However, it would be simple if we could just merge the two datasets together to create one, but we ran into one problem. The airport dataset providing us with climate information contained only 265 airports, while our other dataset contained over 3375 datapoints, which are all the airports in the United States.


\section {Visualization and Choices}

Here is visualization and choices

\section {Feedback}

\section {Feature Contribution}

\section {Challenges}

\section {Work Distribution}

\subsection {Muhammad Alafifi}

\subsection {Jiawei Wu}

\textbf{For our visualization}, I was responsible for gathering and processing all the datasets relating to the airports as well as the climate data on each airport. I was responsible for implementing the time slider and its functionalities, creating the rendering of the US map, plotting the airports in their respective location, selecting colors for their respective climate indicators, and creating the pop-up window that shows the climate data of each airport. I was also responsible for creating the legend that shows the range of values for each climate indicator. In addition, I also implemented a selection for the climate indicators for the airport points for users to switch between temperature, dew point, wind speed, humidity, and air pressure.

\textbf{For our project write up}, I was responsible for writing the abstract, [1] introduction and background, [2a] data collection for the airport data.


% HERE ENDS WHAT IS WRITTEN BY US

\section{Figures and Tables}

Do not place figures and tables at the back of the manuscript. Figures and tables should be placed and sized as they are likely to appear in the final article. 

Figures and Tables should be labelled and referenced in the standard way using the \verb|\label{}| and \verb|\ref{}| commands.

\subsection{Sample Figure}

Figure \ref{fig:false-color} shows an example figure.

\begin{figure}[ht]
\centering
% \fbox{\includegraphics[width=\linewidth]{opticafig1}}
\caption{Dark-field image of a point absorber.}
\label{fig:false-color}
\end{figure}

\subsection{Sample Table}

Table \ref{tab:shape-functions} shows an example table.

\begin{table}[htbp]
\centering
\caption{\bf Shape Functions for Quadratic Line Elements}
\begin{tabular}{ccc}
\hline
local node & $\{N\}_m$ & $\{\Phi_i\}_m$ $(i=x,y,z)$ \\
\hline
$m = 1$ & $L_1(2L_1-1)$ & $\Phi_{i1}$ \\
$m = 2$ & $L_2(2L_2-1)$ & $\Phi_{i2}$ \\
$m = 3$ & $L_3=4L_1L_2$ & $\Phi_{i3}$ \\
\hline
\end{tabular}
  \label{tab:shape-functions}
\end{table}

\section{Sample Equation}

Let $X_1, X_2, \ldots, X_n$ be a sequence of independent and identically distributed random variables with $\text{E}[X_i] = \mu$ and $\text{Var}[X_i] = \sigma^2 < \infty$, and let
\begin{equation}
S_n = \frac{X_1 + X_2 + \cdots + X_n}{n}
      = \frac{1}{n}\sum_{i}^{n} X_i
\label{eq:refname1}
\end{equation}
denote their mean. Then as $n$ approaches infinity, the random variables $\sqrt{n}(S_n - \mu)$ converge in distribution to a normal $\mathcal{N}(0, \sigma^2)$.

\section{Sample Algorithm}

Algorithms can be included using the commands as shown in algorithm \ref{alg:euclid}.

\begin{algorithm}
\caption{Euclid’s algorithm}\label{alg:euclid}
\begin{algorithmic}[1]
\Procedure{Euclid}{$a,b$}\Comment{The g.c.d. of a and b}
\State $r\gets a\bmod b$
\While{$r\not=0$}\Comment{We have the answer if r is 0}
\State $a\gets b$
\State $b\gets r$
\State $r\gets a\bmod b$
\EndWhile\label{euclidendwhile}
\State \textbf{return} $b$\Comment{The gcd is b}
\EndProcedure
\end{algorithmic}
\end{algorithm}

\subsection{Supplementary materials in Optica Publishing Group journals}
Optica Publishing Group journals allow authors to include supplementary materials as integral parts of a manuscript. Such materials are subject to peer-review procedures along with the rest of the paper and should be uploaded and described using the Prism manuscript system. Please refer to the \href{https://opg.optica.org/submit/style/supplementary_materials.cfm}{Author Guidelines for Supplementary Materials in Optica Publishing Group Journals} for more detailed instructions on labeling supplementary materials and your manuscript. For preprints submitted to Optica Open a link to supplemental material should be included in the submission.

\textbf{Authors may also include Supplemental Documents} (PDF documents with expanded descriptions or methods) with the primary manuscript. At this time, supplemental PDF files are not accepted for JOCN or PRJ. To reference the supplementary document, the statement ``See Supplement 1 for supporting content.'' should appear at the bottom of the manuscript (above the References heading). 

\begin{figure}[ht!]
% \centering\includegraphics{opticafig2}
\caption{Terahertz focusing metalens.}
\end{figure}


\subsection{Sample Dataset Citation}

1. M. Partridge, "Spectra evolution during coating," figshare (2014), http://dx.doi.org/10.6084/m9.figshare.1004612.

\subsection{Sample Code Citation}

2. C. Rivers, "Epipy: Python tools for epidemiology," Figshare (2014) [retrieved 13 May 2015], http://dx.doi.org/10.6084/m9.figshare.1005064.

\section{Backmatter}
Backmatter sections should be listed in the order Funding/Acknowledgment/Disclosures/Data Availability Statement/Supplemental Document section. An example of backmatter with each of these sections included is shown below.

\begin{backmatter}
\bmsection{Funding} Content in the funding section will be generated entirely from details submitted to Prism. Authors may add placeholder text in the manuscript to assess length, but any text added to this section in the manuscript will be replaced during production and will display official funder names along with any grant numbers provided. If additional details about a funder are required, they may be added to the Acknowledgments, even if this duplicates information in the funding section. See the example below in Acknowledgements. For preprint submissions, please include funder names and grant numbers in the manuscript.

\bmsection{Acknowledgments} The section title should not follow the numbering scheme of the body of the paper. Additional information crediting individuals who contributed to the work being reported, clarifying who received funding from a particular source, or other information that does not fit the criteria for the funding block may also be included; for example, ``K. Flockhart thanks the National Science Foundation for help identifying collaborators for this work.''

\bmsection{Disclosures} Disclosures should be listed in a separate section at the end of the manuscript. List the Disclosures codes identified on the \href{https://opg.optica.org/submit/review/conflicts-interest-policy.cfm}{Conflict of Interest policy page}. If there are no disclosures, then list ``The authors declare no conflicts of interest.''

\smallskip

\noindent Here are examples of disclosures:


\bmsection{Disclosures} ABC: 123 Corporation (I,E,P), DEF: 456 Corporation (R,S). GHI: 789 Corporation (C).

\bmsection{Disclosures} The authors declare no conflicts of interest.


\bmsection{Data Availability Statement} A Data Availability Statement (DAS) will be required for all submissions beginning 1 March 2021. The DAS should be an unnumbered separate section titled ``Data Availability'' that
immediately follows the Disclosures section. See the \href{https://opg.optica.org/submit/review/data-availability-policy.cfm}{Data Availability Statement policy page} for more information.

There are four common (sometimes overlapping) situations that authors should use as guidance. These are provided as minimal models, and authors should feel free to
include any additional details that may be relevant.



\begin{enumerate}
\item When datasets are included as integral supplementary material in the paper, they must be declared (e.g., as "Dataset 1" following our current supplementary materials policy) and cited in the DAS, and should appear in the references.

\bmsection{Data availability} Data underlying the results presented in this paper are available in Dataset 1, Ref. [3].

\item When datasets are cited but not submitted as integral supplementary material, they must be cited in the DAS and should appear in the references.

\bmsection{Data availability} Data underlying the results presented in this paper are available in Ref. [3].

\item If the data generated or analyzed as part of the research are not publicly available, that should be stated. Authors are encouraged to explain why (e.g.~the data may be restricted for privacy reasons), and how the data might be obtained or accessed in the future.

\bmsection{Data availability} Data underlying the results presented in this paper are not publicly available at this time but may be obtained from the authors upon reasonable request.

\item If no data were generated or analyzed in the presented research, that should be stated.

\bmsection{Data availability} No data were generated or analyzed in the presented research.
\end{enumerate}

\bigskip

\noindent Data availability statements are not required for preprint submissions.

\bmsection{Supplemental document}
See Supplement 1 for supporting content. 

\end{backmatter}

\section{References}

Note that \emph{Optics Letters} and \emph{Optica} short articles use an abbreviated reference style. Citations to journal articles should omit the article title and final page number; this abbreviated reference style is produced automatically when the \emph{Optics Letters} journal option is selected in the template, if you are using a .bib file for your references.

\bigskip
\noindent Add citations manually or use BibTeX. See \cite{Zhang:14,OPTICA,FORSTER2007,testthesis,manga_rao_single_2007}. List up to three author names in references, and if there are more than three authors use \emph{et al.} after that.

% Bibliography
\bibliography{sample}

% Full bibliography added automatically for Optics Letters submissions; the following line will simply be ignored if submitting to other journals.
% Note that this extra page will not count against page length
\bibliographyfullrefs{sample}

%Manual citation list
%\begin{thebibliography}{1}
%\bibitem{Zhang:14}
%Y.~Zhang, S.~Qiao, L.~Sun, Q.~W. Shi, W.~Huang, %L.~Li, and Z.~Yang,
 % \enquote{Photoinduced active terahertz metamaterials with nanostructured
  %vanadium dioxide film deposited by sol-gel method,} Opt. Express \textbf{22},
  %11070--11078 (2014).
%\end{thebibliography}

% Please include bios and photos of all authors for aop articles
\ifthenelse{\equal{\journalref}{aop}}{%
\section*{Author Biographies}
\begingroup
\setlength\intextsep{0pt}
\begin{minipage}[t][6.3cm][t]{1.0\textwidth} % Adjust height [6.3cm] as required for separation of bio photos.
  \begin{wrapfigure}{L}{0.25\textwidth}
    \includegraphics[width=0.25\textwidth]{john_smith.eps}
  \end{wrapfigure}
  \noindent
  {\bfseries John Smith} received his BSc (Mathematics) in 2000 from The University of Maryland. His research interests include lasers and optics.
\end{minipage}
\begin{minipage}{1.0\textwidth}
  \begin{wrapfigure}{L}{0.25\textwidth}
    \includegraphics[width=0.25\textwidth]{alice_smith.eps}
  \end{wrapfigure}
  \noindent
  {\bfseries Alice Smith} also received her BSc (Mathematics) in 2000 from The University of Maryland. Her research interests also include lasers and optics.
\end{minipage}
\endgroup
}{}


\end{document}
